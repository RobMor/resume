%%%%%%%%%%%%%%%%%%%%%%%%%%%%%%%%%%%%%%%%%
% Medium Length Professional CV
% LaTeX Template
% Version 2.0 (8/5/13)
%
% This template has been downloaded from:
% http://www.LaTeXTemplates.com
%
% Original author:
% Trey Hunner (http://www.treyhunner.com/)
%
% Important note:
% This template requires the resume.cls file to be in the same directory as the
% .tex file. The resume.cls file provides the resume style used for structuring the
% document.
%
%%%%%%%%%%%%%%%%%%%%%%%%%%%%%%%%%%%%%%%%%

%----------------------------------------------------------------------------------------
%	PACKAGES AND OTHER DOCUMENT CONFIGURATIONS
%----------------------------------------------------------------------------------------

\documentclass{resume} % Use the custom resume.cls style

\usepackage[left=0.4in,top=0.3in,right=0.4in,bottom=0.3in]{geometry} % Document margins

\usepackage{multicol} % Fancy multi-column stuff

\newenvironment{hanging}{
    \list{}{
        \itemindent=-1em
        \leftmargin=1em
        \topsep=0pt
        \partopsep=0pt
    }
    \item\relax
}{
    \endlist
}
  

\name{Robert Morrison} % Your name
\info{robbieguy98@gmail.com \\ (240) 780-1059} % Your address
\info{\href{https://robmor.dev}{robmor.dev} \\ \href{https://linkedin.com/in/robmorr}{linkedin.com/in/robmorr} \\ \href{https://github.com/robmor}{github.com/robmor}} % Your phone number and email

\begin{document}

%----------------------------------------------------------------------------------------
%	EDUCATION SECTION
%----------------------------------------------------------------------------------------

\begin{rSection}{Education}

{\bf University of Maryland, College Park} \hfill {May 2020} \\
{\em B.S. in Computer Science \& Statistics} \hfill {\em GPA 3.6}

\begin{hanging}
Recognition: {Brian G. Lyons Computer Science Endowed Scholarship (2019)}, {RSA Conference Security Scholars (2019)}, {Maryland Summer Scholars Grant (2018)}
\end{hanging}

\setlength{\multicolsep}{0em} % Set the space after the list title
\begin{multicols}{4}[Relevant Coursework:] % Number of columns and header
    \begin{list}{-}{\leftmargin=1em} % Options for bullet symbol and subsection indentation
    \itemsep -0.5em \vspace{-0.5em} % Compress items in list together for aesthetics
	\item Machine Learning
	\item Probability Theory
	\item Software Engineering
	\item Data Science
	\item Algorithm Design
	\item Comp. Methods
	\item Database Design
	\item Artificial Intelligence
    \end{list}
\end{multicols}

\end{rSection}

%----------------------------------------------------------------------------------------
%	WORK EXPERIENCE SECTION
%----------------------------------------------------------------------------------------

\begin{rSection}{Experience}

%------------------------------------------------

\begin{rSubsection}{Lockheed Martin}{Jun 2019 - Aug 2019}{Software Engineering Intern}{Rochester, NY}
\item Re-designed, built and tested a flight data analysis tool in Python used by data scientists and engineers every day
\item Overhauled the documentation and fielded maintenance and feature requests for the tool
\item Built an internal package management system to boost daily productivity for data scientists, engineers and developers
\item Integrated the package management system directly with continuous integration and deployment software
\item Constructed a proof of concept Kafka cluster visualization tool as a side project
\end{rSubsection}

%------------------------------------------------

\begin{rSubsection}{National Institute of Standards and Technology}{Jun 2018 - Aug 2018}{Scientific Programmer}{Gaithersburg, MD}
\item Developed a web scraping tool to harvest solar panel output and power grid usage from utility websites
\item Pioneered the overhaul of a 10 year old flame speed analysis program written in C++ to update it to modern code standards and make it more maintainable in the future while increasing the speed of calculations
\item Worked with researchers directly to develop tests for the programs and make sure the new methods weren't negatively affecting the accuracy of the results
\item Documented the resulting code thoroughly to ensure the future health of the program
\end{rSubsection}

%------------------------------------------------

\begin{rSubsection}{University of Maryland}{May 2017 - Present}{Undergraduate Researcher}{College Park, MD}
\item Received the competitive Maryland Summer Scholars grant which is given to less than 30 undergraduates each summer
\item Researched and implemented new data oriented vulnerability detection processes for use in the context of software development using Sci-Kit Learn
\item Wrote a full technical report describing experimental design and results along with a poster presenting those findings
\end{rSubsection}

%------------------------------------------------

\begin{rSubsection}{FedCentric Technologies}{Jan 2017 - Aug 2017}{Data Science Intern}{College Park, MD}
\item Rapidly learned data science and machine learning techniques over the course of a semester
\item Replicated and extended past research on machine learning vulnerability detection on old data-sets using a variety of Python machine learning libraries and R data management tools
%\item Analyzed additional data using annotation schemes by applying improved machine learning methods that yielded more consistent results than our previous attempt at analysis
\item Improved predictive capabilities suggesting possible industry applications that would reduce the cost of vulnerability detection by finding problems before release
\end{rSubsection}

%------------------------------------------------

\begin{rSubsection}{Notable Projects}{}{}{}
\item NetZero -- An extensible tool to collect, manage, and analyze several data sources; meant to be used when analyzing the efficiency of a house
\item Kakuro Solver -- A Haskell Kakuro solver built from scratch which can solve a large Kakuro board in under a second
\item Trigger Happy -- A computer vision enabled smart nerf gun that prevents what would be lethal shots; won HackPSU
\end{rSubsection}

%------------------------------------------------

\end{rSection}

%----------------------------------------------------------------------------------------
%	TECHNICAL SKILLS SECTION
%----------------------------------------------------------------------------------------

\begin{rSection}{Technical Skills}

\begin{tabular}{ @{} >{\bfseries}l @{\hspace{6ex}} l }
Programming Languages & Python, C++, C, Java, JavaScript, Haskell, Rust, OCaml, SQL \\
Tools & Numpy, Pandas, Sci-Kit Learn, OpenCV, Git, Linux, CI/CD \\
Other & AutoCAD, \LaTeX, MatLab, Unit Testing, Agile Methodologies
\end{tabular}

\end{rSection}

\end{document}

