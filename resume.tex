%%%%%%%%%%%%%%%%%%%%%%%%%%%%%%%%%%%%%%%%%
% Medium Length Professional CV
% LaTeX Template
% Version 2.0 (8/5/13)
%
% This template has been downloaded from:
% http://www.LaTeXTemplates.com
%
% Original author:
% Trey Hunner (http://www.treyhunner.com/)
%
% Important note:
% This template requires the resume.cls file to be in the same directory as the
% .tex file. The resume.cls file provides the resume style used for structuring the
% document.
%
%%%%%%%%%%%%%%%%%%%%%%%%%%%%%%%%%%%%%%%%%

%----------------------------------------------------------------------------------------
%	PACKAGES AND OTHER DOCUMENT CONFIGURATIONS
%----------------------------------------------------------------------------------------

\documentclass{resume} % Use the custom resume.cls style

\usepackage[left=0.4in,top=0.3in,right=0.4in,bottom=0.3in]{geometry} % Document margins

\usepackage{multicol} % Fancy multi-column stuff

\newenvironment{hanging}{
    \list{}{
        \itemindent=-1em
        \leftmargin=1em
        \topsep=0pt
        \partopsep=0pt
    }
    \item\relax
}{
    \endlist
}
  

\name{Robert Morrison} % Your name
\info{me@robmor.dev \\ (240) 780-1059} % Your address
\info{\href{https://robmor.dev}{robmor.dev} \\ \href{https://linkedin.com/in/robmorr}{linkedin.com/in/robmorr} \\ \href{https://github.com/robmor}{github.com/robmor}} % Your phone number and email

\begin{document}

%----------------------------------------------------------------------------------------
%	EDUCATION SECTION
%----------------------------------------------------------------------------------------

\begin{rSection}{Education}

{\bf University of Maryland, College Park} \hfill {May 2020} \\
{\em B.S. in Computer Science \& Statistics} \hfill {\em GPA 3.6}

\begin{hanging}
Recognition: {Brian G. Lyons Computer Science Endowed Scholarship (2019)}, {Maryland Summer Scholars Grant (2018)}
\end{hanging}

\end{rSection}

%----------------------------------------------------------------------------------------
%	TECHNICAL SKILLS SECTION
%----------------------------------------------------------------------------------------

\begin{rSection}{Technical Skills}

\begin{tabular}{ @{} >{\bfseries}l @{\hspace{6ex}} l }
Programming Languages & Python, C++, C, Java, SQL, Haskell, Rust, JavaScript, Go, OCaml \\
Tools & Docker, Git, Linux, CI/CD, Numpy, Pandas, Sci-Kit Learn \\
\end{tabular}

\end{rSection}

%----------------------------------------------------------------------------------------
%	WORK EXPERIENCE SECTION
%----------------------------------------------------------------------------------------

\begin{rSection}{Professional Experience}

%------------------------------------------------

\begin{rSubsection}{Lockheed Martin}{Jun 2019 - Aug 2019}{Software Engineering Intern}{Rochester, NY}
\item Re-designed, built and tested a flight data analysis tool in Python used by data scientists and engineers every day
\item Overhauled the documentation and fielded maintenance and feature requests for the tool
\item Built an internal package management system to boost daily productivity for data scientists, engineers and developers
\item Integrated the package management system directly with continuous integration and deployment software
\end{rSubsection}

%------------------------------------------------

\begin{rSubsection}{National Institute of Standards and Technology}{Jun 2018 - Aug 2018}{Software Engineer}{Gaithersburg, MD}
\item Overhauled a legacy flame speed analysis program written in C++ to update it to modern code standards and make it more maintainable in the future while increasing the speed of calculations
\item Worked with researchers directly to develop tests for the programs and make sure the new methods were not negatively affecting the accuracy of the results
\item Documented the resulting code thoroughly to ensure the maintainability of the program
\end{rSubsection}

%------------------------------------------------

\begin{rSubsection}{University of Maryland}{May 2017 - Present}{Undergraduate Researcher}{College Park, MD}
\item Researched and implemented new data oriented vulnerability detection processes for use in the context of software development using Sci-Kit Learn
\item Received the competitive Maryland Summer Scholars grant which is given to less than 30 undergraduates each summer
\item Wrote a full technical report describing experimental design and results along with a poster presenting those findings
\end{rSubsection}

%------------------------------------------------

\begin{rSubsection}{FedCentric Technologies}{Jan 2017 - Aug 2017}{Data Science Intern}{College Park, MD}
\item Rapidly learned data science and machine learning techniques over the course of a semester
\item Replicated and extended past research on machine learning vulnerability detection on old data-sets using a variety of Python machine learning libraries and R data management tools
\item Boosted predictive capabilities by 7\% suggesting possible industry applications
\end{rSubsection}

%------------------------------------------------

\end{rSection}

%----------------------------------------------------------------------------------------
%	PUBLICATIONS SECTION
%----------------------------------------------------------------------------------------

\begin{rSection}{Publications}

\begin{hanging}
	Robert Morrison. “Improved Static Vulnerability Detection Methods for use in Large Code Bases.” \textit{RSA Conference 2019: Security Scholars}. 2019.
\end{hanging}
	\vspace{-0.5em} % Jank to get the pubs to take up less space
\begin{hanging}
	Hogan, Kevin, et al. “The Challenges of Labeling Vulnerability Contributing Commits.” \textit{The 4th International Workshop on Reliability and Security Data Analysis}. 2019.
\end{hanging}

\end{rSection}

%----------------------------------------------------------------------------------------
%	PROJECT EXPERIENCE SECTION
%----------------------------------------------------------------------------------------

\begin{rSection}{Project Experience}

\begin{rSubsection}{NetZero}{Apr 2019 - Present}{A tool to glean insights on the efficiency of a house}{}
\item Built, tested, documented and deployed a fully fledged Python command line tool
\item Utilized multiple web API's and SQLite to collect and warehouse data in a fault tolerant manner
\end{rSubsection}

\begin{rSubsection}{Trigger Happy}{Oct 2018}{A smart nerf gun}{}
\item Prevents lethal shots from a nerf gun using TensorFlow, a webcam, and an arduino board
\item Won HackPSU Overall - Grand Prize and Booz Allen Hamilton - Best Machine Learning Hack
\end{rSubsection}

\end{rSection}

\end{document}

